\chapter{Dnevnik promjena dokumentacije}
			
		\begin{longtabu} to \textwidth {|X[2, l]|X[13, l]|X[3, l]|X[3, l]|}
			\hline \multicolumn{1}{|l|}{\textbf{Rev.}}	& \multicolumn{1}{l|}{\textbf{Opis promjene/dodatka}} & \multicolumn{1}{|l|}{\textbf{Autori}} & \multicolumn{1}{l|}{\textbf{Datum}} \\[3pt] \hline
			\endfirsthead
			
			\hline \multicolumn{1}{|l|}{\textbf{Rev.}}	& \multicolumn{1}{l|}{\textbf{Opis promjene/dodatka}} & \multicolumn{1}{|l|}{\textbf{Autori}} & \multicolumn{1}{l|}{\textbf{Datum}} \\[3pt] \hline
			\endhead
			
			\hline 
			\endlastfoot
			
			0.1.0 & Napravljen predložak, ispunjene osnovne informacije i početak pisanja opisa projekta	& Bićanić & 24.10.2019. \\[3pt] \hline
			0.1.1 & Nastavak pisanja opisa projekta & Bićanić & 27.10.2019.		\\[3pt] \hline
			0.1.2 & Završena prva verzija opisa projekta & Bićanić & 28.10.2019. \\[3pt] \hline
			0.2.0 & Započeto pisanje UC dijagrama & Vasilj & 28.10.2019. \\[3pt] \hline
			0.2.1 & Završeno pisanje UC dijagrama & Vasilj & 29.10.2019. \\[3pt] \hline
			0.3.0 & Započeto pisanje Opisa Baze Podataka & Bićanić & 1.11.2019. \\[3pt] \hline
			0.3.1 & Završeno pisanje Opisa Baze Podataka & Bićanić & 3.11.2019. \\[3pt] \hline
			0.3.2 & Dodan dijagram i ER model baze podataka & Bićanić & 3.11.2019. \\[3pt] \hline
		
			
		\end{longtabu}
	
	
		\textit{Moraju postojati glavne revizije dokumenata 1.0 i 2.0 na kraju prvog i drugog ciklusa. Između tih revizija mogu postojati manje revizije već prema tome kako se dokument bude nadopunjavao. Očekuje se da nakon svake značajnije promjene (dodatka, izmjene, uklanjanja dijelova teksta i popratnih grafičkih sadržaja) dokumenta se to zabilježi kao revizija. Npr., revizije unutar prvog ciklusa će imati oznake 0.1, 0.2, …, 0.9, 0.10, 0.11.. sve do konačne revizije prvog ciklusa 1.0. U drugom ciklusu se nastavlja s revizijama 1.1, 1.2, itd.}