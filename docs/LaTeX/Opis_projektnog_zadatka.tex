\chapter{Opis projektnog zadatka}
		
		\textbf{\textit{dio 1. revizije}}\\
		
		Svakodnevno se susrećemo sa neispravno odloženim otpadom, nerijetko upravo zato što su kontejneri predviđeni za njega puni ili čak pretrpani. Jedna od posljedica svega toga je i da recikliranje postaje znatno otežano.
		
		Ovim projektom želimo riješiti taj problem tako što bismo razvili web aplikaciju za cjelokupan sustav odvoza otpada. Kroz nju bi građani na karti mogli slati stanje pojedinog kontejnera, a aplikacija bi bila integrirana sa postojećim komunalnim službama koje bi tim putem dobivale obavijesti o punim kontejnerima, što bi olakšalo planiranje rute i fokusiralo pražnjenje kontejnera na mjesta na kojima je potrebno.
		\linebreak
		
		Aplikacija prilikom pokretanja nudi izbornik kao i polje za unos datuma, te prikazuje kartu područja u okolici klijentovog uređaja na kojoj su označeni kontejneri. Putem izbornika se neregistrirani korisnici mogu registrirati, a neprijavljeni prijaviti. Aplikacija razlikuje tri vrste prijavljenih korisnika (poredano po razini ovlasti, od najniže prema najvišoj):
		\begin{packed_item}
			\item \textit{Građanin}
			\item \textit{Komunalni radnik}
			\item \textit{Administrator}
		\end{packed_item}
	
		Koristeći se oznakama na karti, svi korisnici, uključujući i neregistrirane, mogu za svaki kontejner vidjeti informacije o prijavljivanju i pražnjenju kontejnera, i to za onaj dan koji je upisan u polju za unos datuma.
		\linebreak
		
		Građaninu je za registraciju dovoljna ispravna e-mail adresa i šifra, ali ne mogu postojati dva korisnika sa istom e-mail adresom.
		Nakon izvršene i odobrene prijave i u slučaju da dan upisan u polju datuma odgovara današnjem, građaninu je omogućeno prijavljivanje različitih događaja kontejnera u svojem okruženju pritiskom na neki kontejner: 
		\begin{packed_item}
			\item može kontejner označiti kao pun
			\item može kontejner označiti kao pretrpan
			\item ukoliko primijeti ispražnjen kontejner koji je označen kao pretrpan, može ga označiti ispražnjenim
		\end{packed_item}
		Bez obzira na datum u polju za unos datuma, korisnik može bilo koji kontejner spremiti u osobni popis kontejnera kako bi imao brži pristup kontejnerima u koje češće odlaže otpad (kao što su, primjerice, oni u blizini stana).
		\linebreak
		
		Korisnik u ulozi komunalnog radnika ima sve ovlasti koje ima i građanin. On se ne može registrirati sam, već tu akciju obavlja isključivo korisnik u ulozi administratora. Za komunalnog redara sustav dodatno pamti ime, prezime i Osobni Identifikacijski Broj (OIB). Svaki komunalni radnik je dodjeljen jednom području grada (ili općenito mjesta u kojem se aplikacija koristi), te za sve kontejnere toga područja ima dodatne ovlasti:
		\begin{packed_item}
			\item može isplanirati rutu odvoza otpada (putem izbornika)
			\item može \textbf{svaki} kontejner označiti kao prazan
		\end{packed_item}
		
		Ovlasti administratora sustava uz sve dosad navedene omogućuju i dodatne funkcionalnosti, a njegov korisnički račun se stvara ručno. Uloga administratora je:
		\begin{packed_item}
			\item raspodjela radnika po područjima mjesta na kojem se aplikacija koristi
			\item premještaj radnika, ovisno o trenutnim uvjetima
			\item dodavanje radnika, stvaranje njegovog korisničkog računa
			\item brisanje radnika i pripadajućeg korisničkog računa
			\item dodavanje i brisanje kontejnera
		\end{packed_item}
	
		
		\hline
		
		\begin{packed_item}
			\item \textit{potencijalna korist ovog projekta}
			\item \textit{postojeća slična rješenja (istražiti i ukratko opisati razlike u odnosu na zadani zadatak). Dodajte slike koja predočavaju slična rješenja.}
			\item \textit{skup korisnika koji bi mogao biti zainteresiran za ostvareno rješenje.}
			\item \textit{mogućnost prilagodbe rješenja }
			\item \textit{opseg projektnog zadatka}
			\item \textit{moguće nadogradnje projektnog zadatka}
		\end{packed_item}
		
		\textit{Za pomoć pogledati reference navedene u poglavlju „Popis literature“, a po potrebi konzultirati sadržaj na internetu koji nudi dobre smjernice u tom pogledu.}
		\eject
		
		\section{Primjeri u LaTeXu}
		
		\textit{Ovo potpoglavlje izbrisati.}\\

		U nastavku se nalaze različiti primjeri kako koristiti osnovne funkcionalnosti LaTeXa koje su potrebne za izradu dokumentacije. Za dodatnu pomoć obratiti se asistentu na projektu ili potražiti upute na sljedećim web sjedištima:
		\begin{itemize}
			\item Upute za izradu diplomskog rada u LaTeXu - \url{https://www.fer.unizg.hr/_download/repository/LaTeX-upute.pdf}
			\item LaTeX projekt - \url{https://www.latex-project.org/help/}
			\item StackExchange za Tex - \url{https://tex.stackexchange.com/}\\
		
		\end{itemize} 	


		
		%Ovo poglavlje je potrebno prilikom predaje obrisati
		
		\underbar{podcrtani tekst}, 
		\textbf{podebljani tekst}, 
		\textit{nagnuti tekst}\\
		\normalsize primjer
		\large primjer
		\Large primjer
		\LARGE {primjer}
		\huge {primjer}
		\Huge primjer
		\normalsize
				
		\begin{packed_item}
			
			\item  primjer
			\item  primjer
			\item  primjer
			\item[] \begin{packed_enum}
				
				\item primjer
				\item primjer
			\end{packed_enum}
			
		\end{packed_item}
		
		\noindent primjer url-a: \url{https://www.fer.unizg.hr/predmet/opp/projekt}
		
		
		\begin{longtabu} to \textwidth {|X[8, l]|X[8, l]|X[16, l]|} %definicija sirine polja
			
			\hline \multicolumn{3}{|c|}{\textbf{naslov unutar tablice}}	 \\[3pt] \hline
			\endfirsthead
			
			\hline \multicolumn{3}{|c|}{\textbf{naslov unutar tablice}}	 \\[3pt] \hline
			\endhead
			
			\hline 
			\endlastfoot
			
			\rowcolor{LightGreen}IDKorisnik & INT	&  	Lorem ipsum dolor sit amet, consectetur adipiscing elit, sed do eiusmod  	\\ \hline
			korisnickoIme	& VARCHAR &   	\\ \hline 
			email & VARCHAR &   \\ \hline 
			ime & VARCHAR	&  		\\ \hline 
			\cellcolor{LightBlue} primjer	& VARCHAR &   	\\ \hline 
			
			
		\end{longtabu}
		

		\begin{table}[H]
			
			
			
			\begin{longtabu} to \textwidth {|X[8, l]|X[8, l]|X[16, l]|} %definicija sirine polja
				
				\hline 
				\endfirsthead
				
				\hline 
				\endhead
				
				\hline 
				\endlastfoot
				
				\rowcolor{LightGreen}IDKorisnik & INT	&  	Lorem ipsum dolor sit amet, consectetur adipiscing elit, sed do eiusmod  	\\ \hline
				korisnickoIme	& VARCHAR &   	\\ \hline 
				email & VARCHAR &   \\ \hline 
				ime & VARCHAR	&  		\\ \hline 
				\cellcolor{LightBlue} primjer	& VARCHAR &   	\\ \hline 
				
				
			\end{longtabu}
	
			\caption{\label{tab:referencatablica} Naslov ispod tablice.}
		\end{table}
		
		\begin{figure}[H]
			\includegraphics[scale=0.4]{slike/aktivnost.PNG}
			\centering
			\caption{Primjer slike s potpisom}
			\label{fig:promjene}
		\end{figure}
		
		\begin{figure}[H]
			\includegraphics[width=\linewidth]{slike/aktivnost.PNG}
			\caption{Primjer slike s potpisom 2}
			\label{fig:promjene2}
		\end{figure}
		
		
		
		\eject
		
	