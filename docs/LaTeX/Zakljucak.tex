\chapter{Zaključak i budući rad}
	
	Zadatak naše grupe bio je razvoj web aplikacije za sustav odvoza otpada uz koju bi se građani mogli uključiti te dati svoj doprinos prijavljivanjem stanja kontejnera te također imaju mogućnost pregleda arhive stanja kontejnera. Aplikacija je integrirana sa postojećim komunalnim službama koje tim putem dobivaju obavijesti o punim kontejnerima, što im olakšava planiranje rute i fokusira pražnjenje kontejnera na mjesta na kojima je stvarno potrebno. Nakon 15 tjedana rada u timu i razvoja, ostvarili smo zadani cilj. Sama provedba projekta tekla je u dvije etape.
	
		Prva etapa projekta uključivala je okupljanje tima za razvoj aplikacije, dodjelu projektnog zadatka, raspodjelu zadataka i intenzivan rad na dokumentiranju zahtjeva. Kvalitetna provedba prve faze uvelike je olakšala daljnji rad pri realizaciji osmišljenog sustava. Izrađeni obrasci i dijagrami (obrasci uporabe, sekvencijski dijagrami, model baze podataka, dijagram razreda) bili su od pomoći podtimovima zaduženima za razvoj backenda i frontenda. 
		
		Druga etapa projekta, iako trajanjem neznatno kraća od prve, bila je daleko intenzivnija i zahtjevnija od prve što po pitanju samostalnog rada članova što po zahtjevnosti samih zadataka. Manjak iskustva članova u izradi sličnih implementacijskih rješenja primorao je članove na samostalno učenje odabranih alata i programskih jezika kako bi ispunili dogovorene ciljeve. Osim realizacije rješenja, u drugoj fazi je bilo potrebno dokumentirati ostale UML dijagrame i izraditi popratnu dokumentaciju kako bi budući korisnici mogli lakše koristiti ili vršiti preinake na sustavu. Dobro izrađen kostur projekta uštedio nam je mnogo vremena prilikom izrade aplikacije te smo izbjegli moguće pogreške u izradi koje bi bile vremenski skupe za ispravljanje u daljnjoj fazi projekta.
		
		Komunikacija među članovima tima odvijala se putem Whatsappa kako bismo postigli što bolju i nadasve pravovremenu informiranost svih članova. Moguće proširenje postojeće inačice sustava svakako je izrada mobilne aplikacije čime bi cilj projektnog zadatka bio ostvaren u većoj mjeri, odnosno bio bi mnogo pristupaćniji i praktičniji za korištenje no s web aplikacijom. 
		
		Sudjelovanje na ovom projektu bilo je iznimno vrijedno iskustvo zbog same prirode zadatka, odnosno odlične prilike za učenjem novih tehnologija, ali dakako i zbog iskustva koje smo stekli intenzivnim timskim radom kroz ovih nekoliko tjedana. Projekt nas je naučio važnosti dobre vremenske raspodjele i koordiniranosti među članovima koji rade na projektu. Vidimo iznimno velik potencijal za unaprijeđivanje naše aplikacije i samog tima, ali obzirom na nedostatak vremena i iskustva veoma smo zadovoljni postignutim te svim stećenim znanjima i iskustvima.
		
		\eject 